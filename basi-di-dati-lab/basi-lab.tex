\documentclass[a4paper, 10pt]{article}
\usepackage[italian]{babel}
\usepackage[T1]{fontenc}
\usepackage[utf8]{inputenc}
\usepackage{amsmath}
\usepackage{amsfonts}
\usepackage{amsthm}
\usepackage{graphicx}
\usepackage[sans]{frontespizio}
\usepackage{hyperref}
\hypersetup{hidelinks,
	colorlinks = true,
	urlcolor = black, 
	linkcolor = black}
\usepackage[T1]{fontenc}
\usepackage[utf8]{inputenc}
\usepackage[margin=3cm]{geometry}
\usepackage{multicol}
\usepackage{booktabs}
\usepackage{fancyhdr}
\usepackage{tikz}
\usetikzlibrary{calc}
\usepackage{listings}
\lstset{language = SQL,
	basicstyle=\ttfamily,
	showstringspaces=false}
\usepackage{fancyvrb}
\pagestyle{fancy}

\begin{document}
	\begin{frontespizio}
		\Preambolo{\usepackage{datetime}}
		\Istituzione{Università degli Studi di Verona}
		\Divisione{Dipartimento di informatica}
		\Facolta{Scienze e Ingegneria}
		\Scuola{Laurea in Informatica}
		\Titolo{Basi di Dati}
		\Sottotitolo{Programma di laboratorio}
		\Candidato{Davide Bianchi}
		\NCandidato{Autori}
		\Annoaccademico{2016/2017}
	\end{frontespizio}
	
	\tableofcontents
	
	\newpage
	
	\section{Gestione base di dati con Postgresql}
	Di seguito si trova una panoramica dei comandi postgres più comuni per la gestione di una base di dati.
	
	\subsection{Comando CREATE TABLE}
	Il comando \lstinline{CREATE TABLE} è usato per creare tabelle nella base di dati.
	La sintassi generale è:
	\begin{lstlisting}
CREATE TABLE nomeTabella (
	nomeAttributo dominioAttributo vincoli,
	...
);
	\end{lstlisting}
	dove \lstinline|nomeAttributo| è il nome dell'attributo nella tabella, \lstinline|dominioAttributo| è il dominio dell'attributo da aggiungere alla tabella.
	
	
	
\end{document}