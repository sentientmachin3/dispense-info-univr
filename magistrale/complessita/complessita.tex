\documentclass[a4paper]{article}
\usepackage[italian]{babel}
\usepackage[T1]{fontenc}
\usepackage[utf8]{inputenc}
\usepackage{csquotes}
\renewcommand{\mkbegdispquote}[2]{\itshape}
\usepackage{palatino}
\usepackage{frontespizio}
\usepackage{mathtools, nccmath}
\usepackage{amsmath}
\usepackage{amsfonts}
\usepackage{amsthm}
\usepackage{amssymb}
\usepackage[margin=3cm]{geometry}
\usepackage{booktabs}
\usepackage{fancyhdr}
\pagestyle{fancy}
\lhead{\nouppercase{\leftmark}}
\rhead{\nouppercase{\rightmark}}
\chead{}
\lfoot{}
\cfoot{\thepage}
\rfoot{}
\renewcommand{\headrulewidth}{0.4pt}

\usepackage{hyperref}
\usepackage{xcolor}
\hypersetup{
	hidelinks, 
	colorlinks = true,
	linkcolor = black,
}
\usepackage[OT1]{eulervm}

\begin{document}
	\begin{frontespizio}
		\Universita{Verona}
		\Dipartimento{Informatica}
		\Scuola{Laurea Magistrale in Ingegneria e Scienze Informatiche}
		\Titoletto{Algoritmi}
		\Titolo{Complessità}
		\Sottotitolo{Riassunto dei principali argomenti}
		\Candidato{Davide Bianchi}
		\Candidato{Matteo Danzi}
		\Annoaccademico{2017/2018}
		\NCandidato{Autori}
	\end{frontespizio}
	
	\tableofcontents
	
	\newpage

	\section{Introduzione}
	
	\subsection{Cos'è la complessità computazionale}
		Nella teoria della complessità ci si pone la seguente domanda:
		
		\begin{displayquote}
		Come scalano le risorse necessarie per risolvere un problema all'aumentare delle dimensioni del problema?
		\end{displayquote} 

		La teoria della \textit{complessità computazionale} è una parte dell’informatica teorica che si
		occupa principalmente di classificare i problemi in base alla quantità di \textit{risorse computazionali} (come il tempo di calcolo e lo spazio di memoria) che essi richiedono per
		essere risolti. Tale quantità è detta anche \textit{costo computazionale} del problema.
		
	\subsection{Problemi \textit{facili} e \textit{difficili}}
		Vediamo quattro esempi di problemi che classificheremo come facili o difficili:
		\begin{enumerate}
			\item (\textbf{Eulerian Cycle}) Esiste un modo per attraversare ogni arco di un grafo una e una sola volta?
			
			Il problema si può vedere anche nella forma più piccola del problema dei \textit{sette ponti di Königsberg}:
			
			A Königsberg ci sono 7 ponti, esiste un percorso che attraversa tutti i ponti una e una sola volta per poi tornare al punto di partenza?
			\item (\textbf{Hamiltonian Cycle}) Esiste un modo per attraversare ogni nodo di un grafo una e una sola volta? 
			\item N è un numero primo?
			\item Quali sono i fattori primi di un numero?
		\end{enumerate}
		
		

\end{document}
