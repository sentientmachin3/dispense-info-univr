\documentclass[a4paper, 11pt]{article}
\usepackage[english]{babel}
\usepackage[utf8]{inputenc}
\usepackage{amsmath}
\usepackage{graphicx}
\usepackage{float}
\usepackage{fixltx2e}
\usepackage{listings}
\usepackage{color}
\usepackage{latexsym}
\usepackage{lstautogobble}
\usepackage[colorinlistoftodos]{todonotes}
\usepackage[margin=3cm]{geometry}
\usepackage{hyperref}
\usepackage{libertine}
\usepackage{tikz}
\hypersetup{
	hidelinks, 
	colorlinks = true,
	linkcolor = black,
}

\usetikzlibrary{shapes, arrows}

\newtheorem{definit}{Definizione}[subsection]

\begin{document}
	\clearpage
	\begin{titlepage}
		\centering
		\vspace*{\fill}
		{\scshape\LARGE Università degli Studi di Verona \par}
		\vspace{1.5cm}
		\line(1,0){230} \\
		{\huge\bfseries Sicurezza delle reti\par}
		\line(1,0){230} \\
		\vspace{0.5cm}
		{\scshape\Large Riassunto dei principali argomenti\par}
		\vspace{2cm}
		{\Large\itshape Davide Bianchi\par}
		\vspace{1cm}
		\vspace{5cm}
		\vspace*{\fill}
		% Bottom of the page
		{\large \today\par}
	\end{titlepage}
	\thispagestyle{empty}
	\newpage
	\tableofcontents
	\newpage
	
	\section{Introduzione}
	\subsection{Spam di definizioni}
	
	\begin{definit}[Information Security]
		Protezione delle informazioni e dei sistemi per impedirne l'accesso non autorizzato, uso, divulgazione, modifica o distruzione.
	\end{definit}
	
	\begin{definit}[Network Security]
		Protezione dell'accesso a risorse situate all'interno di una rete.
	\end{definit}
	

\end{document}