\documentclass[a4paper, 11pt]{article}
\usepackage[english]{babel}
\usepackage[utf8]{inputenc}
\usepackage[T1]{fontenc}
\usepackage{amsmath}
\usepackage[colorinlistoftodos]{todonotes}
\usepackage[margin=3cm]{geometry}
\usepackage{libertine}
\usepackage{hyperref}
\hypersetup{
	hidelinks, 
	colorlinks = true,
	linkcolor = black,
}


\begin{document}
 \clearpage
 \begin{titlepage}
 	\centering
 	\vspace*{\fill}
 	{\scshape\LARGE Università degli Studi di Verona \par}
 	\vspace{1.5cm}
 	\line(1,0){280} \\
 	{\huge\bfseries Intelligenza Artificiale\par}
 	\line(1,0){280} \\
 	\vspace{0.5cm}
 	{\scshape\Large Riassunto dei principali argomenti\par}
 	\vspace{2cm}
 	{\Large\itshape Davide Bianchi\par}
 	\vspace{1cm}

 	\vspace{5cm}
 	\vspace*{\fill}
 	% Bottom of the page
 	{\large \today\par}
 \end{titlepage}
 \thispagestyle{empty}
\newpage
\tableofcontents
\newpage


\section{Agenti razionali}
\paragraph{Agenti.}
Un agente è semplicemente un'entità che riceve percezioni e produce una risposta con delle azioni. Formalmente un agente è una funzione \[ f:P^\ast \to A \] dove $P^\ast$ è lo storico delle percezioni e $A$ è un insieme di azioni.

Notare che se un agente ha $\vert P \vert$ possibili percezioni in ingresso, dopo $T$ unità di tempo la funzione agente avrà il seguente numero di entries: \[ \sum_{t=1}^{T} \vert P \vert^t  \]

Un agente è in generale una struttura formata da un'architettura fisica e un programma, e prende in input una percezione attuale e ritorna in output l'azione successiva da svolgere. 

Esistono principalmente 4 tipi di agenti: \begin{itemize}
	\item agenti \textit{simple-reflex};
	\item agenti \textit{reflex} con stato;
	\item agenti \textit{goal-based};
	\item agenti \textit{utility-based}.
\end{itemize}

\paragraph{Performace measure.} La \textit{performance-measure} costituisce una sorta di punteggio che misura il comportamento dell'agente nell'ambiente in cui opera. Quindi, data una performance measure e le percezioni attuale dell'agente, questo sceglie la sequenze di azioni che la massimizzano.

\paragraph{Ambienti.} Un ambiente, ovvero lo spazio in cui l'agente opera, è caratterizzato dai seguenti tratti: \begin{itemize}
	\item Osservabilità;
	\item Determinismo;
	\item Episodicità;
	\item Staticità;
	\item Discretezza;
	\item Presenza di altri agenti.
\end{itemize}

Il tipo di ambiente povviamente condiziona il design degli agenti che vi operano.

\end{document}