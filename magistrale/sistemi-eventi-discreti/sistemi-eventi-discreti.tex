\documentclass[a4paper]{article}
\usepackage[italian]{babel}
\usepackage[T1]{fontenc}
\usepackage[utf8]{inputenc}
\usepackage{frontespizio}
\usepackage{amsmath}
\usepackage{amsfonts}
\usepackage{amsthm}

\usepackage{cleveref}

\usepackage{fancyhdr}
\pagestyle{fancy}
\lhead{\nouppercase{\leftmark}}
\rhead{\nouppercase{\rightmark}}
\chead{}
\lfoot{}
\cfoot{\thepage}
\rfoot{}
\renewcommand{\headrulewidth}{0.4pt}

\usepackage{hyperref}
\hypersetup{
	hidelinks, 
	colorlinks = true,
	linkcolor = black,
}

\theoremstyle{definition}
\newtheorem{exmp}{Esempio}[section]
\newtheorem{defn}{Definizione}[section]


\begin{document}
	\begin{frontespizio}
		\Universita{Verona}
		\Dipartimento{Informatica}
		\Scuola{Laurea Magistrale in Ingegneria e Scienze Informatiche}
		\Titoletto{}
		\Titolo{Sistemi ad Eventi Discreti}
		\Sottotitolo{Riassunto dei principali argomenti}
		\Candidato{Chiunque voglia scriverla}
		\Annoaccademico{2017/2018}
		\NCandidato{Autore}
	\end{frontespizio}
	
	\tableofcontents
	
	\newpage
	
\end{document}