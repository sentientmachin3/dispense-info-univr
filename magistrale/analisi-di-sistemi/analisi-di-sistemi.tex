\documentclass[a4paper, 11pt]{article}
\usepackage[italian]{babel}
\usepackage[T1]{fontenc}
\usepackage[utf8]{inputenc}
\usepackage{amsmath}
\usepackage{amsfonts}
\usepackage{amsthm}
\usepackage{frontespizio}
\usepackage{hyperref}
\hypersetup{hidelinks,
	colorlinks = true,
	urlcolor = black, 
	linkcolor = black}
\usepackage[margin=3cm]{geometry}
\usepackage{booktabs}
\usepackage{fancyhdr}
\usepackage{listings}
\usepackage{stmaryrd}
\usepackage[strict]{changepage}
\usepackage{galois}

\newcommand{\parts}[1]{\mathcal{P}(#1)}
\newcommand{\galoistuple}{\langle C, \alpha, \gamma , A \rangle}

\newtheorem{definit}{Definizione}[subsection]


\begin{document}
	\begin{frontespizio}
		\Preambolo{\usepackage{datetime}}
		\Istituzione{Università degli Studi di Verona}
		\Divisione{Dipartimento di informatica}
		\Titolo{Analisi di Sistemi informatici}
		\Scuola{}
		\Sottotitolo{Riassunto dei principali argomenti}
		\Candidato{Davide Bianchi}
		\NCandidato{Autore}
		\Annoaccademico{2017/2018}
	\end{frontespizio}
	
	\tableofcontents
	\newpage
	
	\section{Introduzione}
	Argomenti contenuti:
	\begin{itemize}
		\item Interpretazione astratta
		\item Analisi statica
		\item Analisi dinamica
	\end{itemize}

	\section{Preliminari matematici}
	\subsection{Ordini parziali}
	\subsection{Reticoli}
	\subsection{Teoremi di punto fisso}
	
	\section{Interpretazione astratta}
	\subsection{Introduzione}
	Lo scopo è quello di trovare un'approssimazione di una semantica $\langle P \rangle$ di $\llbracket P \rrbracket$ tale per cui valgano:
	\begin{itemize}
		\item \textit{correttezza:} $\llbracket P \rrbracket \subseteq \langle P \rangle$;
		\item \textit{decidibilità:} $\langle P \rangle \subseteq Q$ è decidibile ($Q$ è un insieme di semantiche che soddisfa la proprietà di interesse).
	\end{itemize}

	Se entrambe le proprietà sono soddisfatte, allora vale che \[ (\langle P \rangle \subseteq Q) \Rightarrow (\llbracket P \rrbracket \subseteq Q) \]
	
	La semantica è data da una coppia $\langle D, f \rangle$ dove $D$ è una coppia $\langle D, \leq_D$ rappresentante un dominio semantico e $f: D \to D$ è una funzione di trasferimento con una soluzione a punto fisso.
	
	Dato un oggetto concreto, definiamo:
	\begin{itemize}
		\item un \textbf{oggetto astratto} come una rappresentazione matematica sovra-approssimata del corrispondente concreto;
		\item un \textbf{dominio astratto} come un insieme di oggetti astratti con delle operazioni astratte, che approssimano quelle concrete;
		\item una funzione di \textbf{astrazione} $\alpha$ che mappa oggetti concreti in oggetti astratti;
		\item una funzione di \textbf{concretizzazione} $\gamma$ che mappa oggetti astratti in oggetti concreti.
	\end{itemize}
	
	La caratteristica peculiare delle astrazioni è che solo alcune proprietà vengono osservate con esattezza, le altre vengono solo approssimate. In sostanza, dato un dominio astratto $A$, gli elementi di $A$ sono osservati con esattezza, gli altri sono approssimati o l'informazione è persa del tutto.
	
	\paragraph{Proprietà.} L'insieme delle proprietà $\parts{\Sigma}$ di oggetti in $\Sigma$ è l'insieme di elementi che gode di quella proprietà. Questo insieme di proprietà costituisce un reticolo completo \[ \langle \parts{\Sigma}, \subseteq, \emptyset, \cup, \cap, \neg \rangle \] dove:
	\begin{itemize}
		\item $\subseteq$ è l'implicazione logica;
		\item $\Sigma$ è \verb|true|;
		\item $\cup$ è la disgiunzione (oggetti che godono di $P$ o di $Q$ appartengono a $P \cup Q$);
		\item $\cap$ è la congiunzione (oggetti che godono di $P$ e di $Q$ appartengono a $P \cap Q$);
		\item $\neg$ è la negazione (oggetti che non godono di $P$ stanno in $\Sigma \setminus P$).
	\end{itemize}

	\paragraph{Direzione dell'astrazione.}
	Quando si approssima una proprietà concreta $P \in \parts{\Sigma}$ usando una proprietà astratta $\overline{P}$, deve essere stabilito un criterio per definire quando $\overline{P}$ è un'approssimazione di $P$.
	
	Si distinguono quindi i seguenti casi:
	\begin{itemize}
		\item approssimazione \textit{da sopra}: $P \subseteq \overline{P}$;
		\item approssimazione \textit{da sotto}: $P \supseteq \overline{P}$.
	\end{itemize}
	
	Dato un oggetto $o$, si vuole quindi sapere se $o \in P$:
	\begin{align*}
		P \supseteq \overline{P}: \begin{cases}
			\text{"Si"} &o \in \overline{P} \\
			\text{"Non lo so"} &o \notin  \overline{P}
		\end{cases} \qquad
		P \subseteq \overline{P}: \begin{cases}
		\text{"No"} &o \notin \overline{P} \\
		\text{"Non lo so"} &o \in \overline{P}\\
		\end{cases}
	\end{align*} 
	
	\paragraph{Migliore approssimazione.}
	Definiamo come \textit{migliore approssimazione} di una proprietà $P$ in $A$ il glb delle over-approximation di $P$ in $A$, ossia: \[  \overline{P} = \bigcap \lbrace \overline{P'} \in A | P \subseteq \overline{P'} \rbrace \in A \]
	
	\subsection{Connessione di Galois}
	Imponiamo il vincolo che $\alpha$ e $\gamma$ siano monotone, allora concludiamo che: \begin{itemize}
		\item $\gamma \circ \alpha: C \to C$ è \textbf{estensiva}: $\gamma(\alpha(c)) \geq c$;
		\item $\alpha \circ \gamma : A \to A$ è \textbf{riduttiva}: $\alpha(\gamma(a)) \leq a$.
	\end{itemize}

	Le definizioni qui sopra dicono rispettivamente che:
	\begin{itemize}
		\item $\alpha$ perde informazione, e $\gamma$ non la può recuperare;
		\item $\gamma$ non perde informazione.
	\end{itemize}

	\begin{definit}
		Dati due poset $\langle A , \leq_A \rangle$ e $\langle C , \leq_C \rangle$, e due funzioni monotone $\alpha: C \to A$ e $\gamma: A \to C$, diciamo che $\galoistuple$ è una connessione di Galois se:
		\begin{itemize}
			\item $\forall c \in \mathcal{C}: c \leq_C \gamma(\alpha(c))$
			\item $\forall a \in \mathcal{A}: \alpha(\gamma(a)) \leq_A a$
		\end{itemize}
		
		Se inoltre vale che $\forall a \in \mathcal{A}: \alpha(\gamma(a)) = a$, allora $\galoistuple$ è un'inserzione di Galois.
	\end{definit}
	Una connessione e un'inserzione di Galois sono rappresentate rispettivamente come \[  C \galois{\alpha}{\gamma} A \qquad C \galoiS{\alpha}{\gamma} A \]
	La funzione $\alpha$ è detta \textit{aggiunta sinistra}, mentre la funzione $\gamma$ è detta \textit{aggiunta destra}.
	
	\begin{definit}
		Data una connessione di Galois $ C \galois{\alpha}{\gamma} A$, sono equivalenti:
		\begin{itemize}
			\item $C \galoiS{\alpha}{\gamma} A$;
			\item $\alpha$ è suriettiva;
			\item $\gamma$ è iniettiva.
		\end{itemize}
	\end{definit}

	Inoltre, dati due domini astratti, non esistono due coppie $(\alpha, \gamma)$ che formino una connessione di Galois; quindi la connessione di Galois tra due domini è \textbf{unica}, e le funzioni sono identificabili attraverso:
	\begin{align*}
		\alpha(c) &= \bigwedge \lbrace a \in A \vert c \leq_C \gamma(a) \rbrace \\
		\gamma(a) &= \bigvee \lbrace c \in C \vert \alpha(c) \leq_A a \rbrace
	\end{align*}
	
\end{document}