\documentclass[a4paper, 11pt]{article}
\usepackage[italian]{babel}
\usepackage[T1]{fontenc}
\usepackage[utf8]{inputenc}
\usepackage{amsmath}
\usepackage{amsfonts}
\usepackage{amsthm}
\usepackage{frontespizio}
\usepackage{hyperref}
\hypersetup{hidelinks,
	colorlinks = true,
	urlcolor = black, 
	linkcolor = black}
\usepackage[margin=3cm]{geometry}
\usepackage{booktabs}
\usepackage{fancyhdr}
\usepackage{listings}
\usepackage{stmaryrd}
\usepackage[strict]{changepage}


\begin{document}
	\begin{frontespizio}
		\Preambolo{\usepackage{datetime}}
		\Istituzione{Università degli Studi di Verona}
		\Divisione{Dipartimento di informatica}
		\Titolo{Analisi di Sistemi informatici}
		\Scuola{}
		\Sottotitolo{Riassunto dei principali argomenti}
		\Candidato{Davide Bianchi}
		\NCandidato{Autore}
		\Annoaccademico{2017/2018}
	\end{frontespizio}
	
	\tableofcontents
	\newpage
	
	\section{Introduzione}
	Argomenti contenuti:
	\begin{itemize}
		\item Interpretazione astratta
		\item Analisi statica
		\item Analisi dinamica
	\end{itemize}

	\section{Preliminari matematici}
	\subsection{Ordini parziali}
	\subsection{Reticoli}
	\subsection{Teoremi di punto fisso}
	
	\section{Interpretazione astratta}
	Lo scopo è quello di trovare un'approssimazione di una semantica $\langle P \rangle$ di $\llbracket P \rrbracket$ tale per cui valgano:
	\begin{itemize}
		\item \textit{correttezza:} $\llbracket P \rrbracket \subseteq \langle P \rangle$;
		\item \textit{decidibilità:} $\langle P \rangle \subseteq Q$ è decidibile ($Q$ è un insieme di semantiche che soddisfa la proprietà di interesse).
	\end{itemize}

	Se entrambe le proprietà sono soddisfatte, allora vale che \[ (\langle P \rangle \subseteq Q) \Rightarrow (\llbracket P \rrbracket \subseteq Q) \]
	
	La semantica è data da una coppia $\langle D, f \rangle$ dove $D$ è una coppia $\langle D, \leq_D$ rappresentante un dominio semantico e $f: D \to D$ è una funzione di trasferimento con una soluzione a punto fisso.
	
	Dato un oggetto concreto, definiamo:
	\begin{itemize}
		\item un \textbf{oggetto astratto} come una rappresentazione matematica sovra-approssimata del corrispondente concreto;
		\item un \textbf{dominio astratto} come un insieme di oggetti astratti con delle operazioni astratte, che approssimano quelle concrete;
		\item una funzione di \textbf{astrazione} $\alpha$ che mappa oggetti concreti in oggetti astratti;
		\item una funzione di \textbf{concretizzazione} $\gamma$ che mappa oggetti astratti in oggetti concreti.
	\end{itemize}
	
	La caratteristica peculiare delle astrazioni è che solo alcune proprietà vengono osservate con esattezza, le altre vengono solo approssimate. In sostanza, dato un dominio astratto $A$, gli elementi di $A$ sono osservati con esattezza, gli altri sono approssimati o l'informazione è persa del tutto.
	
	\paragraph{Proprietà.} L'insieme delle proprietà $P(\Sigma)$ di oggetti in $\Sigma$ è l'insieme di elementi che gode di quella proprietà. Questo insieme di proprietà costituisce un reticolo completo \[ \langle P(\Sigma), \subseteq, \emptyset, \cup, \cap, \neg \rangle \] dove:
	\begin{itemize}
		\item $\subseteq$ è l'implicazione logica;
		\item $\Sigma$ è \verb|true|;
		\item $\cup$ è la disgiunzione (oggetti che godono di $P$ o di $Q$ appartengono a $P \cup Q$);
		\item $\cap$ è la congiunzione (oggetti che godono di $P$ e di $Q$ appartengono a $P \cap Q$);
		\item $\neg$ è la negazione (oggetti che non godono di $P$ stanno in $\Sigma \setminus P$).
	\end{itemize}
	

	
\end{document}