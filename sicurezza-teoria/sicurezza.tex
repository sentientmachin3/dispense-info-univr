\documentclass[a4paper, 11pt]{article}
\usepackage[italian]{babel}
\usepackage{sans}
\usepackage[T1]{fontenc}
\usepackage[utf8]{inputenc}
\usepackage{frontespizio}
\usepackage{fancyhdr}
\usepackage{txfonts}
\usepackage[margin=3cm]{geometry}
\usepackage{fancyvrb}
\usepackage{multicol}
\usepackage{hyperref}
\hypersetup{hidelinks,
	colorlinks = true,
	urlcolor = black, 
	linkcolor = black}

\pagestyle{fancy}
\lhead{\nouppercase{\leftmark}}
\rhead{\nouppercase{\rightmark}}
\chead{}
\lfoot{}
\cfoot{\thepage}
\rfoot{}

\begin{document}
	\begin{frontespizio}
		\Istituzione{Università degli Studi di Verona}
		\Divisione{Dipartimento di informatica}
		\Scuola{Ultima modifica: \today }
		\Titolo{Sicurezza di Rete}
		\Sottotitolo{Riassunto dei principali argomenti}
		\Candidato{Davide Bianchi}
		\Candidato{Matteo Danzi}
		\NCandidato{Autori}
		\Annoaccademico{2016/2017}
	\end{frontespizio}
	
	\tableofcontents
	
	\newpage
	

	\begin{multicols}{2}
		\begin{abstract}
			Questa dispensa è scritta per la parte teorica del corso di Programmazione e sicurezza di Rete. Il codice \LaTeX è disponibile a \url{https://github.com/alx79/dispense-info-univr.git}
		\end{abstract}
		\section{Introduzione}
		Il fatto di garantire una protezione a determinati assets implica il garantire di alcune proprietà:
		\begin{enumerate}
			\item Confidenzialità: un utente non dovrebbe venire a conoscenza di cose che non è autorizzato a conoscere (riservatezza dei dati, privacy);
			
			\item Disponibilità: rendere disponibili ad un utente autorizzato le informazioni che può avere e che richiede;
			
			\item Integrità: impedire l'alterazione di dati e informazioni in maniera diretta o indiretta (anche in seguito a incidenti);
			
			\item Autenticità: ad un utente deve essere garantita l'autenticità delle informazioni che riceve;
			
			\item Tracciabilità: le azioni di un utente devono essere tracciate in modo univoco, in modo evitare eventuali casi di ripudiablità.
		\end{enumerate}
		
		Ciò che può compromette le caratteristiche sopra elencate sono le minacce e gli attacchi. Viene definita \textit{minaccia} una possibile violazione della sicurezza, mentre invece un \textit{attacco} è una violazione effettiva della sicurezza.
		
		Gli attacchi possono essere sostanzialmente di 4 tipologie:
		\begin{itemize}
			\item \textit{attivi}: tentativi di alterare il funzionamento di un sistema;
			\item \textit{passivi:} tentativi di carpire informazioni senza intaccare i meccanismi del sistema;
			\item \textit{interni:} effettuati da un'entità interna al sistema;
			\item \textit{esterni:} effettuati da un'entità esterna al sistema.
		\end{itemize}
		
		Gli attacchi (o le minacce) sono suddivisi in classi:
		\begin{itemize}
			\item \textit{disclosure:} accesso non autorizzato alle informazioni;
			\item \textit{deception:} accettazione di dati falsi;
			\item \textit{disruption:} interruzione o prevenzione di informazioni corrette;
			\item \textit{usurpation:} controllo non autorizzato di alcune parti del sistema.
		\end{itemize}
		
		\section{Crittografia e integrità}
		Lo scopo storico della crittografia è quello di garantire la privacy, ossia come garantire che un'informazione ricevuta provenga effettivamente dall'utente che ci si aspetta l'abbia mandata.
		
		\subsection{Funzioni Hash}
		Una funzione hash è una funzione che trasforma un messaggio di lunghezza arbitraria in uno di lunghezza fissa (viene chiamato \textit{hash} o \textit{digest} del messaggio originale). Le funzioni hash attualmente più utilizzate sono MD5 e SHA.
		
		Per soddisfare le condizioni di sicurezza, gli algoritmi che gestiscono le funzioni hash dovrebbero avere le seguenti caratteristiche:
		\begin{itemize}
			\item \textit{coerenti}: a input uguali corrispondono output uguali;
			\item \textit{casuali}: per impedire l'interpretazione del messaggio originale;
			\item \textit{univoci}: la probabilità che due messaggi generino due hash uguali deve essere remota;
			\item \textit{non invertibili}: deve essere impossibile (o computazionalmente complesso) risalire dal digest al messaggio originale.
		\end{itemize}
	
		Le funzioni hash vengono anche usate come fingerprint per verificare che nessuno sia intervenuto sul messaggio originale (altrimenti i due digest sarebbero diversi, vedi esempio).
		
		Ora daremo un esempio di come possa avvenire una comunicazione sfruttando le funzioni hash.
		Alcune definizioni:
		\begin{itemize}
			\item \textit{m} è il messaggio in chiaro;
			\item $H(m)$ è l'hash del messaggio;
			\item $c(x)$ è la funzione di cifratura;
			\item \textit{A} e \textit{B} sono due utenti.
		\end{itemize}
		Indichiamo inoltre come $H_A(m)$ l'hash del messaggio scritto da \textit{A}.
		
		\paragraph{Esempio.} \textit{A} scrive un messaggio e ne utilizza il testo come input di una funzione di hash, che genera il digest $H_A(m)$. \textit{A} poi manda $c(m + H_A(m))$ a \textit{B}. 
		
		\textit{B} decifra e separa il contenuto del messaggio cifrato che ha ricevuto, e calcola con la funzione di hash un hash denominato $H_B(m)$. Se vale \[ H_B(m) = H_A(m) \] il messaggio è autentico.
		
		Se i due utenti non sono interessati a mantenere occultato il messaggio, viene utilizzato un \textit{MAC} (Message Authentication Code), un segreto condiviso conosciuto da entrambi gli utenti. In questo caso viene mandato al destinatario il pacchetto con \[ m + H_A(m + s) \] 
		
		Usando un MAC si ha anche garanzia di autenticità, grazie al segreto condiviso. Qui sorge un nuovo problema: come poter scambiare con l'altro utente un segreto condiviso su un canale protetto? Per ovviare a questo problema è stato proposto un meccanismo di \textit{firma digitale}, che \textbf{non usa chiavi segrete}.
		
		
		
		
		
		
	
	
	
	
	
	
	
	
	
	
	
	
	
	\end{multicols}
\end{document}